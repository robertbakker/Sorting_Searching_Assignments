\documentclass[12pt,notitlepage]{article}
\usepackage[utf8]{inputenc}
\usepackage{natbib}
\usepackage[dutch]{babel}
\usepackage{graphicx}

\usepackage{blindtext}
\usepackage[final]{pdfpages}
\usepackage[a4paper, total={6.3in,9in},footskip=76pt]{geometry}


% make section start with Alphabet letters
\renewcommand{\thesection}{\alph{section}}

% plots
\usepackage{pgfplots}
\pgfplotsset{compat=1.8}
\pgfplotsset{mystyle/.style={%
        width=6in,
        ylabel={Sorting time in nanoseconds},
        xlabel={Aantal studenten},
        xmin=0,xmax=160000,
        scaled y ticks = false,
        scaled x ticks = false,	
       	x tick label style={rotate=45, anchor=north east, inner sep=0mm},
       	legend pos=north west
        }}


\usepackage{titlesec}
\titlespacing\section{0pt}{12pt plus 4pt minus 2pt}{0pt plus 2pt minus 2pt}
\titlespacing\subsection{0pt}{0pt plus 4pt minus 2pt}{0pt plus 2pt minus 2pt}
\titlespacing\subsubsection{0pt}{12pt plus 4pt minus 2pt}{0pt plus 2pt minus 2pt}

% turn off word breaking
\usepackage{hyphenat}
\hyphenpenalty 10000
\exhyphenpenalty 10000
\raggedright

% make paragraphs with line spacing
\setlength{\parskip}{\baselineskip}%
\setlength{\parindent}{0pt}%


% Set up code listings for Java
\usepackage{listings}
\usepackage{color}

\definecolor{dkgreen}{rgb}{0,0.6,0}
\definecolor{gray}{rgb}{0.5,0.5,0.5}
\definecolor{mauve}{rgb}{0.58,0,0.82}

\lstset{frame=tb,
  language=Java,
  aboveskip=3mm,
  belowskip=3mm,
  showstringspaces=false,
  columns=flexible,
  basicstyle={\small\ttfamily},
  numbers=none,
  numberstyle=\tiny\color{gray},
  keywordstyle=\color{blue},
  commentstyle=\color{dkgreen},
  stringstyle=\color{mauve},
  breaklines=true,
  breakatwhitespace=true,
  tabsize=3
}

\begin{document}


%----------------------------------------------------------------------------------------
%	ONDERZOEKSVOORSTEL
%----------------------------------------------------------------------------------------

\begin{titlepage}

\newcommand{\HRule}{\rule{\linewidth}{0.5mm}}

\center % Center everything on the page

%----------------------------------------------------------------------------------------
%	HEADING SECTIONS
%----------------------------------------------------------------------------------------
\begin{figure}[h!]
\centering
\includegraphics[scale=0.5]{hva-logo.png}
\end{figure}
\textsc{\LARGE Hogeschool van Amsterdam}\\[1.5cm] % Name of your university/college
\textsc{\Large Sorting \& Searching}\\[0.5cm] % Major heading such as course name
\textsc{\large Efficiëntie van geavanceerde sorteeralgoritmes}\\[0.5cm] % Minor heading such as course title

%----------------------------------------------------------------------------------------
%	TITLE SECTION
%----------------------------------------------------------------------------------------

\HRule \\[0.4cm]
{ \huge \bfseries Practicum 1}\\[0.2cm] % Title of your document
\HRule \\[1.2cm]

% Optimalisatie van gegevensverwerking voor iLocate

%----------------------------------------------------------------------------------------
%	AUTHOR SECTION
%----------------------------------------------------------------------------------------

\begin{minipage}{0.4\textwidth}
\begin{flushleft} \large
\emph{Author:}\\
Robert \textsc{Bakker} % Your name
\linebreak
\linebreak
\emph{Studentnummer:}\\
500689284
\linebreak
\linebreak
\emph{Klas:}\\
IVSE4

\end{flushleft}
\end{minipage}
~
\begin{minipage}{0.4\textwidth}
\begin{flushright} \large
\emph{Author:}\\
Robert \textsc{Bakker} % Your name
\linebreak
\linebreak
\emph{Studentnummer:}\\
500689284
\linebreak
\linebreak
\emph{Klas:}\\
IVSE4
\end{flushright}
\end{minipage}\\[4cm]

%----------------------------------------------------------------------------------------
%	DATE SECTION
%----------------------------------------------------------------------------------------

{\large Blok 2, 2016 - 2017}\\[3cm] % Date, change the \today to a set date if you want to be precise

%----------------------------------------------------------------------------------------
%	LOGO SECTION
%----------------------------------------------------------------------------------------

%\includegraphics{Logo}\\[1cm] % Include a department/university logo - this will require the graphicx package

%----------------------------------------------------------------------------------------

\vfill % Fill the rest of the page with whitespace

\end{titlepage}


%----------------------------------------------------------------------------------------
%	TABLE OF CONTENTS
%----------------------------------------------------------------------------------------
\renewcommand{\contentsname}{Inhoudsopgave}
\tableofcontents
\clearpage

%----------------------------------------------------------------------------------------
%	Opdracht a
%----------------------------------------------------------------------------------------

\section{Resultaten van studenten sorteren met een advanced sort}
\subsection{Advanced sort toevoegen}

\begin{lstlisting}

    // De quicksort accepteert een lijst van objecten met een comparable
    // interface, het beginpunt van links, en het beginpunt van rechts
    private void quicksort(Comparable[] list, int low, int high) {

        // Neem het middelpunt van de array als spil (draaipunt)
        Comparable pivot = list[low + (high - low) / 2];

        int i = low; // linkerkant
        int j = high; // rechterkant

        while (i <= j) {
            // Wanneer object vanaf links kleiner is dan de spil
            // Verschuif naar de volgende in de linkerlijst
            while (list[i].compareTo(pivot) < 0) i++;

            // Wanneer object vanaf rechts groter is dan de spil
            // Verschuif naar de volgende in de rechterlijst
            while (list[j].compareTo(pivot) > 0) j--;

            // Als er een index van de linkerlijst is gevonden, met een waarde
            // die groter is dan de spil, en een index in de rechterlijst met
            // een waarde die kleiner is dan de spil, moeten de 2 waarden
            // worden omgedraaid
            if (i <= j) {
                Comparable temp = list[i];
                list[i] = list[j];
                list[j] = temp;
                i++;
                j--;
            }
        }
        // Hetzelfde voor de rest van de linkerlijst
        if (low < j) {
            quicksort(list, low, j);
        }
        // en voor de rechterlijst
        if (high > i) {
            quicksort(list, i, high);
        }
    }
\end{lstlisting}

\subsection{Efficiëntie - Experiment}


\begin{tikzpicture}
  \begin{axis}[mystyle]
  \addlegendentry{$Quicksort$}
\addplot+[smooth] coordinates
{(10000, 32418821) (20000, 72854465) (40000, 152722902) (80000, 349821005) (160000, 770648150) };

  \end{axis}
\end{tikzpicture}

\subsection{Big O}

2n(log(n))
%----------------------------------------------------------------------------------------
%	Opdracht b
%----------------------------------------------------------------------------------------
\section{Verbetering toevoegen aan algoritme}

\subsection{Experiment}
\begin{tikzpicture}
  \begin{axis}[mystyle]
  \addlegendentry{$Quicksort\ met\ Median$}
\addplot+[smooth] coordinates
{(10000, 59946551) (20000, 83454751) (40000, 156851911) (80000, 354702260) (160000, 795273813) };

  \end{axis}
\end{tikzpicture}
\subsection{Big O}


%----------------------------------------------------------------------------------------
%	Opdracht c
%----------------------------------------------------------------------------------------
\section{Resultaten in een Binary Search Tree en implementatie van rank()}

De input is een lijst van studenten bestaand uit een cijfer en studentnummer. De input set bestaat uit 10000 studenten. Op deze set  voeren wij de rank method uit. om te tellen welke hoeveel studenten lager dan een bepaald cijfer hebben behaald. \\


\subsection{BST implementatie rank}

\begin{lstlisting}

 private int rank(Key key, Node x) {
        if (x == null) {
            return 0;
        }
        int cmp = key.compareTo(x.key);
        if (cmp < 0) {
            return rank(key, x.left);
        } else if (cmp > 0) {
            return x.val.size() + size(x.left) + rank(key, x.right);
        } else {
            return size(x.left);
        }
    }
\end{lstlisting}
\subsection{output rank}
Grade: 1, rank: 0 \\
Grade: 2, rank: 1111 \\
Grade: 3, rank: 2154 \\
Grade: 4, rank: 3297 \\
Grade: 5, rank: 4471 \\
Grade: 6, rank: 5564 \\
Grade: 7, rank: 6713 \\
Grade: 8, rank: 7826 \\
Grade: 9, rank: 8900 \\
Grade: 10, rank: 10000

\end{document}

\end{document}