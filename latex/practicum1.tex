\documentclass[12pt,notitlepage]{article}
\usepackage[utf8]{inputenc}
\usepackage{natbib}
\usepackage[dutch]{babel}
\usepackage{graphicx}

\usepackage{blindtext}
\usepackage[final]{pdfpages}
\usepackage[a4paper, total={6.3in,9in},footskip=76pt]{geometry}


% make section start with Alphabet letters
\renewcommand{\thesection}{\alph{section}}


\usepackage{titlesec}
\titlespacing\section{0pt}{12pt plus 4pt minus 2pt}{0pt plus 2pt minus 2pt}
\titlespacing\subsection{0pt}{0pt plus 4pt minus 2pt}{0pt plus 2pt minus 2pt}
\titlespacing\subsubsection{0pt}{12pt plus 4pt minus 2pt}{0pt plus 2pt minus 2pt}

% turn off word breaking
\usepackage{hyphenat}
\hyphenpenalty 10000
\exhyphenpenalty 10000
\raggedright

% make paragraphs with line spacing
\setlength{\parskip}{\baselineskip}%
\setlength{\parindent}{0pt}%


% Set up code listings for Java
\usepackage{listings}
\usepackage{color}

\definecolor{dkgreen}{rgb}{0,0.6,0}
\definecolor{gray}{rgb}{0.5,0.5,0.5}
\definecolor{mauve}{rgb}{0.58,0,0.82}

\lstset{frame=tb,
  language=Java,
  aboveskip=3mm,
  belowskip=3mm,
  showstringspaces=false,
  columns=flexible,
  basicstyle={\small\ttfamily},
  numbers=none,
  numberstyle=\tiny\color{gray},
  keywordstyle=\color{blue},
  commentstyle=\color{dkgreen},
  stringstyle=\color{mauve},
  breaklines=true,
  breakatwhitespace=true,
  tabsize=3
}



%--- MAKE NEW SUBSUBSUBSECTION COMMAND---------------------------------------------------
\usepackage{titlesec}
\usepackage{hyperref}
\titleclass{\subsubsubsection}{straight}[\subsection]

\newcounter{subsubsubsection}[subsubsection]
\renewcommand\thesubsubsubsection{\thesubsubsection.\arabic{subsubsubsection}}
\renewcommand\theparagraph{\thesubsubsubsection.\arabic{paragraph}} % optional; useful if paragraphs are to be numbered

\titleformat{\subsubsubsection}
  {\normalfont\normalsize\bfseries}{\thesubsubsubsection}{1em}{}
\titlespacing*{\subsubsubsection}
{0pt}{3.25ex plus 1ex minus .2ex}{1.5ex plus .2ex}

\makeatletter
\renewcommand\paragraph{\@startsection{paragraph}{5}{\z@}%
  {3.25ex \@plus1ex \@minus.2ex}%
  {-1em}%
  {\normalfont\normalsize\bfseries}}
\renewcommand\subparagraph{\@startsection{subparagraph}{6}{\parindent}%
  {3.25ex \@plus1ex \@minus .2ex}%
  {-1em}%
  {\normalfont\normalsize\bfseries}}
\def\toclevel@subsubsubsection{4}
\def\toclevel@paragraph{5}
\def\toclevel@paragraph{6}
\def\l@subsubsubsection{\@dottedtocline{4}{7em}{4em}}
\def\l@paragraph{\@dottedtocline{5}{10em}{5em}}
\def\l@subparagraph{\@dottedtocline{6}{14em}{6em}}
\makeatother

\setcounter{secnumdepth}{4}
\setcounter{tocdepth}{4}
%----------------------------------------------------------------------------------------

\begin{document}


%----------------------------------------------------------------------------------------
%	ONDERZOEKSVOORSTEL
%----------------------------------------------------------------------------------------

\begin{titlepage}

\newcommand{\HRule}{\rule{\linewidth}{0.5mm}}

\center % Center everything on the page

%----------------------------------------------------------------------------------------
%	HEADING SECTIONS
%----------------------------------------------------------------------------------------
\begin{figure}[h!]
\centering
\includegraphics[scale=0.5]{hva-logo.png}
\end{figure}
\textsc{\LARGE Hogeschool van Amsterdam}\\[1.5cm] % Name of your university/college
\textsc{\Large Sorting \& Searching}\\[0.5cm] % Major heading such as course name
\textsc{\large Efficiëntie van geavanceerde sorteeralgoritmes}\\[0.5cm] % Minor heading such as course title

%----------------------------------------------------------------------------------------
%	TITLE SECTION
%----------------------------------------------------------------------------------------

\HRule \\[0.4cm]
{ \huge \bfseries Practicum 1}\\[0.2cm] % Title of your document
\HRule \\[1.2cm]

% Optimalisatie van gegevensverwerking voor iLocate

%----------------------------------------------------------------------------------------
%	AUTHOR SECTION
%----------------------------------------------------------------------------------------

\begin{minipage}{0.4\textwidth}
\begin{flushleft} \large
\emph{Author:}\\
Robert \textsc{Bakker} % Your name
\linebreak
\linebreak
\emph{Studentnummer:}\\
500689284
\linebreak
\linebreak
\emph{Klas:}\\
IVSE4

\end{flushleft}
\end{minipage}
~
\begin{minipage}{0.4\textwidth}
\begin{flushright} \large
\emph{Author:}\\
Robert \textsc{Bakker} % Your name
\linebreak
\linebreak
\emph{Studentnummer:}\\
500689284
\linebreak
\linebreak
\emph{Klas:}\\
IVSE4
\end{flushright}
\end{minipage}\\[4cm]

%----------------------------------------------------------------------------------------
%	DATE SECTION
%----------------------------------------------------------------------------------------

{\large Blok 2, 2016 - 2017}\\[3cm] % Date, change the \today to a set date if you want to be precise

%----------------------------------------------------------------------------------------
%	LOGO SECTION
%----------------------------------------------------------------------------------------

%\includegraphics{Logo}\\[1cm] % Include a department/university logo - this will require the graphicx package

%----------------------------------------------------------------------------------------

\vfill % Fill the rest of the page with whitespace

\end{titlepage}


%----------------------------------------------------------------------------------------
%	TABLE OF CONTENTS
%----------------------------------------------------------------------------------------
\renewcommand{\contentsname}{Inhoudsopgave}
\tableofcontents
\clearpage

%----------------------------------------------------------------------------------------
%	Opdracht a
%----------------------------------------------------------------------------------------

\section{Resultaten van studenten sorteren met een advanced sort}


\begin{lstlisting}
// Hello.java
import javax.swing.JApplet;
import java.awt.Graphics;

public class Hello extends JApplet {
    public void paintComponent(Graphics g) {
        g.drawString("Hello, world!", 65, 95);
    }
}
\end{lstlisting}


\subsection{Advanced sort toevoegen}
\subsection{Efficiëntie}

%----------------------------------------------------------------------------------------
%	Opdracht b
%----------------------------------------------------------------------------------------
\section{Verbetering toevoegen aan algoritme}

%----------------------------------------------------------------------------------------
%	Opdracht c
%----------------------------------------------------------------------------------------
\section{Resultaten in een Binary Search Tree en implementatie van rank()}

\end{document}